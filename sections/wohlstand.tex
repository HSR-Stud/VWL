\section{Wohlstand}
\subsection{Preismechanismus und Marktwirtschaft \ho{Ho2}}
\subsubsection{Ausgangslage \ho{Ho2 F4}}
Unbeschr�nkte Bed�rfnisse vs. knappe Ressourcen.
\subsubsection{Opportunit�tskosten \ho{Ho2 F4-6}}
  Kosten, die bei einer Entscheidung anfallen, dass die Vorteile einer
  Handlungsalternative nicht realisiert werden k�nnen.\\
  Beispiel: Weil wir studieren verdienen wir derzeit nichts/weniger. Das Geld
  das wir dadurch verlieren sind die Opportunit�tskosten.\\
  Die Opportunit�tskosten zeigen die Knappheit eines Gutes an.
\subsubsection{Zentrale Rolle der Preise in der Marktwirtschaft \ho{Ho2 F8-9}}
  Der Preis wird von der Nachfrage bestimmt und hat Signalwirkung. Steigt er so
  ist das ein anzeichen f�r knappheit. Planwirtschaft zerst�rt das alles.
\subsubsection{Mikro�konomisches Grundmodell \ho{Ho2 F10-12}}
  \begin{tabular}{ll}
    G�termarkt & Preis - Menge\\
    Arbeitsmarkt & Lohn - Arbeit\\
    Kapitalmarkt & Zins - Kapital\\
  \end{tabular}
\subsubsection{Mindestpreis \ho{Ho2 F16-19}}
  Auswirkungen wenn man einen Mindestpreis einf�hrt:\\
  Konsum sinkt $\Rightarrow$ Angebot steigt $\Rightarrow$ Ware wird exportiert
  oder vergammelt $\Rightarrow$ Konsumentenrente sinkt $\Rightarrow$
  Produzentenrente steigt $\Rightarrow$ Gesamtrente sinkt.\\
  Bei H�chstpreisen geschieht das ganze analog. Man sollte also nicht mit
  Mindest/H�chstpreisen in den Markt eingreifen (auch bei L�hnen!).
  
\subsection{Internationale Arbeitsteilung \ho{Ho3}}
\subsubsection{Komparativer Kostenvorteil \ho{Ho3 F6}}
  Vorgehen:
  \begin{itemize}
    \item Opportunit�tskosten Berechnen f�r alle Artikel.
    \item Opportunit�tskosten vergleichen.
    \item Der Ort mit den tiefsten Opportunit�tskosten produziert und
    exportiert zu einem Preis zwischen Opportunit�tskosten im
    Exportort und den Opportunit�tskosten im Importort (nat�rlich nur ohne
    Z�lle, Transportkosten etc.).
  \end{itemize}
  Falls Gleichstand ist findet kein Handel statt.
\subsubsection{Weltmarktpreis \ho{Ho3 F8-13}}
Hoher Weltmarktpreis: Konsument verliert, Allgemeinheit und Produktion gewinnt.\\
Tiefer Weltmarktpreis bewirkt genau das Gegenteil.\\
Internationale Arbeitsteilung bringt positive Wohlfahrtseffekte unabh�ngig
davon, ob der Weltmarktpreis h�her oder tiefer als der Heimmarktpreis ist.
\subsubsection{Z�lle \ho{Ho3 F15-17}}
\subsubsection{Regionale Integrationsr�ume \ho{Ho3 F22FF}}
