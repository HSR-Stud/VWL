\section{Wohlstand}
\subsection{Preismechanismus und Marktwirtschaft}
\subsubsection{Ausgangslage}
Unbeschr�nkte Bed�rfnisse vs. knappe Ressourcen.
\subsubsection{Opportunit�tskosten}
  Kosten, die bei einer Entscheidung anfallen, dass die Vorteile einer
  Handlungsalternative nicht realisiert werden k�nnen.\\
  Beispiel: Weil wir studieren verdienen wir derzeit nichts/weniger. Das Geld
  das wir dadurch verlieren sind teil der Opportunit�tskosten.\\
  Die Opportunit�tskosten zeigen die Knappheit eines Gutes an.
\subsubsection{Zentrale Rolle der Preise in der Marktwirtschaft}
  Der Preis wird von der Nachfrage bestimmt und hat Signalwirkung. Steigt er so
  ist das ein anzeichen f�r Knappheit. Planwirtschaft zerst�rt das alles.
\begin{multicols}{2}
	\subsubsection{Mikro�konomisches Grundmodell}
	  \begin{tabular}{ll}
	    G�termarkt & Preis - Menge\\
	    Arbeitsmarkt & Lohn - Arbeit\\
	    Kapitalmarkt & Zins - Kapital\\
	  \end{tabular}
	  \includegraphics[width=8cm]{./bilder/h02f15.png}
	\subsubsection{Mindestpreis}
	  Auswirkungen wenn man einen Mindestpreis einf�hrt:\\
	  Konsum sinkt $\Rightarrow$ Angebot steigt $\Rightarrow$ Ware wird exportiert
	  oder vergammelt $\Rightarrow$ Konsumentenrente sinkt $\Rightarrow$
	  Produzentenrente steigt $\Rightarrow$ Gesamtrente sinkt.\\
	  Bei H�chstpreisen geschieht das ganze analog. Man sollte also nicht mit
	  Mindest/H�chstpreisen in den Markt eingreifen (auch bei L�hnen!).
	  \includegraphics[width=8cm]{./bilder/h02f19.png}
\end{multicols}

\subsection{Internationale Arbeitsteilung}
\subsubsection{Komparativer Kostenvorteil}
\begin{multicols}{2}
  Vorgehen:
  \begin{itemize}
    \item Opportunit�tskosten Berechnen f�r alle Artikel.
    \item Opportunit�tskosten vergleichen.
    \item Der Ort mit den tiefsten Opportunit�tskosten produziert und
    exportiert zu einem Preis zwischen Opportunit�tskosten im
    Exportort und den Opportunit�tskosten im Importort (nat�rlich nur ohne
    Z�lle, Transportkosten etc.).
  \end{itemize}
  Falls Gleichstand ist findet kein Handel statt.
  \includegraphics[width=9cm]{./bilder/h03f07.png}
\end{multicols}
\subsubsection{Weltmarktpreis}
Hoher Weltmarktpreis: Konsument verliert, Allgemeinheit und Produktion gewinnt.\\
Tiefer Weltmarktpreis bewirkt genau das Gegenteil.\\
Internationale Arbeitsteilung bringt positive Wohlfahrtseffekte unabh�ngig
davon, ob der Weltmarktpreis h�her oder tiefer als der Heimmarktpreis ist.
\begin{multicols}{2}
	\includegraphics[width=9cm]{./bilder/h03f11.png}
	\includegraphics[width=9cm]{./bilder/h03f13.png}
\end{multicols}

\subsubsection{Z�lle}
\subsubsection{Regionale Integrationsr�ume}
\begin{multicols}{2}
\subsection{Monopolmacht und Wettbewerb}
  Grenzkosten = Kosten f�r letztes produziertes Gut (zunehmend)\\
  Grenzertrag = Ertrag f�r letztes produziertes Gut (abnehmend)\\
  Bei einem Monopol sind die beiden gleich!
  
\subsubsection{Nat�rliche und nicht-nat�rliche Monopole}
  Ein typisches nat�rliches Monopol ist der Eisenbahnverkehr. Man kann nicht
  einer anderen Firma sagen sie soll ein zweites Gleis daneben bauen. Um das
  nat�rliche Monopol zu lockern kann man zum Beispiel die Schienen
  Verstaatlichen aber darauf Verkehr f�r mehrere Firmen zulassen (SBB, DB, SOB,
  BLS etc.). Anderes Beispiel: Strommarkt/netz
  \includegraphics[width=9.5cm]{./bilder/h04f11.png}
\end{multicols}

\newpage
\subsection{Externe Effekte und die Umwelt}
\begin{multicols}{2}
%\begin{tabular}{ll}
  \includegraphics[width=9cm]{./bilder/h05f12.png} %& 
  \includegraphics[width=9cm]{./bilder/h05f17.png} %\\
%\end{tabular}
\end{multicols}

\begin{multicols}{2}
\subsection{Langfristiges Wachstum \ho{Ho2}}
Konjunkturverlauf schwankt mehr als das langfristige Wachstum.
	\includegraphics[width=9cm]{./bilder/h06f06.png}
\subsubsection{Technisches Know How \ho{Ho2 F10,14}}
  Technisches Know How ist der wichtigste Wachstumsfaktor, da er der Einzige ist
  der unbeschr�nkt verf�gbar ist.
  	\includegraphics[width=9cm]{./bilder/h06f07.png}
\end{multicols}
\subsubsection{Patentschutz \ho{Ho2 F15}}
  Patentschutz bietet einen Investitionsschutz, behindert aber technischen
  Fortschritt. Daher muss man dazwischen ein Mittel suchen. 
\subsubsection{Wachstumspolitik in der Schweiz \ho{Ho2 F18-20}}
Wachstumsflaute 2004--2007. Ursachen:
\begin{itemize}
	\item Problem der Schweizer Hochpreisinsel
	\item H�heres Wachstum der Schweizer Staatsquote als andere OECD-L�nder
\end{itemize}
Handlungsfelder der Wachstumspolitik 2012--2015:
\begin{itemize}
	\item Wettbewerb beleben (Wettbewerspolitik)
	\item Wirtschaftliche �ffnung nach aussen (Aussenwirtschaftspolitik)
	\item Hohe Erwerbsbeteiligung (Arbeitsmarktpolitik)
	\item Bildung, Forschung, Innovation st�rken (Bildungs- und
		Arbeitsmarktpolitik)
	\item Gesunde �ffentliche Finanzen (Finanzpolitik)
	\item Rechtliches Umfeld zum F�rdern von Unternehmen (Rechtsetzung)
	\item Tragbare Umweltbeanspruchung (Umweltpolitik)
\end{itemize}
