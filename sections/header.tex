%Angaben zum Dokument
\title{VWL Zusammenfassung}
\newcommand*\modurl{https://github.com/HSR-Stud/VWL/}
\newcommand*\mydate{\today}

\newcommand{\ho}[1]{$_{{\mbox{\small{#1}}}}$}


%Schriftgroesse, Layout, Papierformat, Gleichungen Linksbuendig
\documentclass[10pt,twoside,a4paper,fleqn]{article}

%Abmessungen Layout
\usepackage[left=1cm,right=1cm,top=1cm,bottom=1cm,includeheadfoot]{geometry}

%Package fuer Umlaute
\usepackage[latin1]{inputenc}
\usepackage[T1]{fontenc}
\usepackage[latin1]{inputenc}

%Package fuer Deutsch
\usepackage[ngerman]{babel}

%Formeln
\usepackage{amsmath}
%Spezielle Symbole in Formeln
\usepackage{amssymb}
%Sonderzeichen
\usepackage{textcomp}
%Bilder und Grafiken
\usepackage{graphicx}
%Textfarben
\usepackage{color}
%Fliesstext
\usepackage{wrapfig}
%Header und Footer
\usepackage{fancyhdr}
%Teile des Dokuments Mehrspaltig
\usepackage{multicol}
%Rotieren von Elementen (Bilder, Tabellen...)
\usepackage{rotating}
%Schriftart
\usepackage[scaled]{helvet}
\renewcommand*\familydefault{\sfdefault}

%Autoren aus CONTRIBUTORS.txt laden
\input{texutils/authorsfromfile.tex}
\authorsfromfile{CONTRIBUTORS.txt}{\url{\modurl}}

% Capture title and author
\makeatletter
\let\Title\@title
\let\Author\@author
\normalfont\sbox0{\normalfont\Title\qquad\qquad\Author}
\ifdim\wd0>\textwidth\def\Author{\modurl}\fi
\makeatother

%Seitenaufbau
\pagestyle{fancy} %eigener Seitenstil

%Linker Header
\lhead{\Title}
%Rechter Header
\rhead{\Author}
%Linker Footer
\lfoot{\modurl}
%Center Footer
\cfoot{\thepage}
%Rechter Footer
\rfoot{\mydate}
%Text Linksbuendig, unregelmaessiger rechter Rand
\raggedright
